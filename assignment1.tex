\documentclass[12pt]{article}


\usepackage[margin=1in]{geometry}
\usepackage{amsmath}
\usepackage{amsfonts}
\usepackage{graphicx}      % include this line if your document contains figures
\usepackage{enumerate}
\usepackage{cite}
\usepackage{circuitikz}
\usetikzlibrary{patterns}
\usetikzlibrary{shapes.geometric}


\definecolor{msblue}{rgb}{0.357, 0.608, 0.835}
\definecolor{msgreen}{rgb}{0.439, 0.678, 0.278}
\definecolor{msred}{rgb}{0.753, 0, 0}
\definecolor{msyellow}{rgb}{0.984, 0.737, 0.0196}



\title{Assignment 1 - Models and Representations}
\date{\today}
%\author{Arne Dankers}


\begin{document}
\maketitle

\begin{enumerate}
\item Read sections 1.1, 1.2 and 1.3 in the book ``Introduction to Mathematical Systems Theory", by Jan Polderman and Jan Willems. Based on the text, briefly answer the following questions:
\begin{enumerate}
    \item How does Willems view a model?
    \item What is the central object specifying a mathematical model?
    \item What are the three components of Willems' 'modelling language'?
    \item What is the behavior of a model?
    \item What are latent variables?
    \item What differentiates a Dynamical System from "other" systems?
\end{enumerate}
\item Unit 1 contained many concepts. Create an infographic (i.e. a diagram, illustration, drawing, chart, etc.) with the following concepts on it:
\begin{itemize}
\item Systems in the real world
\item Models of systems
\item Time domain response of a system (response due to initial conditions, step response, response to any input)
\item Representations of a system (differential equations, transfer function, state-space equations, frequency domain)
\item Characteristics of a system (settling time, maximum overshoot, rise time, stability)
\item Key concepts (poles, zeros)
\item Plots (time domain response, s-plane, bode)
\end{itemize}
The key is that your infographic should show/illustrate the relationships between all these elements. 
\item Find the region in the s-plane for the poles of a second order system that meets the following design requirements:
\begin{enumerate}
    \item Peak overshoot $\le 10$\%,
    \item 2\% settling time $\le 5$ seconds,
    \item Peak time $\le 0.5$ seconds. 
\end{enumerate}
\item Select $3$ different transfer functions that satisfy the design requirements. Plot the step responses of these three transfer functions. Plot them all on the same graph. Make sure you include a title, axis labels, and a legend on your plot. Verify that they do in fact meet the design requirements. You can use the functions in \texttt{enel441\_utilities.py} or functions from the Python \texttt{control} package to generate your plot.
\end{enumerate}


\end{document}